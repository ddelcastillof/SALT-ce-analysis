% University of Washington Thesis Template

\documentclass[12pt]{report}
\usepackage[utf8]{inputenc}
\usepackage[margin=1in]{geometry}
\usepackage{setspace}
\usepackage{titling}
% Specifying the font
\usepackage{fontspec}
\setmainfont{Times New Roman}

% Document information - Fill in your details here
\newcommand{\thesistitle}{Cost-effectiveness analysis of a household salt substitute intervention on reducing community-wide blood pressure in Northern Peru}
\newcommand{\authorname}{Darwin Del Castillo Fernández}
\newcommand{\degreename}{Master in Public Health}
\newcommand{\graduationyear}{2025}
\newcommand{\chairname}{Rachel Nugent}
\newcommand{\chairdepartment}{Department of Global Health}
\newcommand{\committeememberone}{Monisha Sharma}
\newcommand{\departmentname}{Global Health}

% Document settings
\setlength{\droptitle}{0pt}
\doublespacing

\begin{document}

%----------------------------------------------------------------------------------------
%	TITLE PAGE
%----------------------------------------------------------------------------------------

\begin{titlepage}
\begin{center}

\vspace*{1cm}
{\LARGE \thesistitle}\\
\vspace{1.5cm}

{\large \authorname}\\
\vspace{1.5cm}

A thesis\\
submitted in partial fulfillment of the\\
requirements for the degree of\\
\vspace{0.5cm}

{\large \degreename}\\
\vspace{0.5cm}

University of Washington\\
\vspace{0.5cm}

{\large \graduationyear}\\
\vspace{1cm}

Committee:\\
\chairname\\
\committeememberone\\
\vspace{1cm}

Program Authorized to Offer Degree:\\
\departmentname\\

\end{center}
\end{titlepage}

%----------------------------------------------------------------------------------------
%	COPYRIGHT PAGE
%----------------------------------------------------------------------------------------

\thispagestyle{empty}
\vspace*{3cm}
\begin{center}
©Copyright \graduationyear\\
\authorname
\end{center}
\clearpage

%----------------------------------------------------------------------------------------
%	ABSTRACT PAGE
%----------------------------------------------------------------------------------------

\thispagestyle{plain}
\begin{center}
University of Washington\\
\vspace{0.5cm}
{\large Abstract}\\
\vspace{0.5cm}
{\large \thesistitle}\\
\vspace{0.5cm}
{\large \authorname}\\
\vspace{0.5cm}
Chair of the Supervisory Committee:\\
\chairname\\
\chairdepartment\\
\end{center}

\vspace{1cm}

% Abstract text starts here
\noindent\textbf{[Background]}

Hypertension is widely recognized as the leading risk factor for cardiovascular diseases, causing over 10 million cardiovascular disease-related deaths [1]. Reducing salt consumption is one of the most cost-effective approaches to lowering blood pressure [2]. A large meta-analysis showed that modest salt intake reductions decrease systolic and diastolic blood pressure among hypertensive and normotensive populations [3]. Despite the evidence, population-level behavioral interventions have shown inconsistent effects on reducing salt intake, indicating that education and awareness-raising alone are insufficient to reduce salt intake at the population level [4].

A growing number of countries have implemented national salt-reduction strategies. Salt substitution can support these efforts, especially in countries where salt is added to cooking and at the table is the primary source of salt intake [5]. This is particularly important in low- and middle-income countries experiencing rising rates of hypertension [6]. Some of these salt substitution strategies have succeeded in countries such as China [7,8], where a study demonstrated cost-effectiveness in reducing stroke and cardiovascular events after five years of follow-up [7]. However, the intervention did not include a component to ensure adherence among participants. The present study aims to evaluate the cost-effectiveness of a household salt substitution intervention at the population level, plus a social marketing campaign among people responsible for food preparation at home in six communities in Northern Peru [9].

The rationale for introducing a social marketing campaign to a salt substitution intervention stems from the common challenge of achieving adherence among participants. To address this, a social marketing campaign was integrated into the intervention design to improve acceptance [10,11]. This addition to the design can affect the cost-effectiveness of the intervention. Therefore, our objective is to capture this potential difference by including the cost of the social marketing campaign in the economic evaluation.

\vspace{0.5cm}
\noindent\textbf{[Study objectives and specific aims]}

This study aims to evaluate the cost-effectiveness of a complex intervention that included a salt substitute accompanied by a social marketing campaign at the population level among women responsible for food preparation at home in six communities in northern Peru. This study will inform decisions at the central government level to propose and scale up salt substitute intervention strategies to reduce the burden of hypertension and related cardiovascular morbidity and mortality in Peru.

\end{document}